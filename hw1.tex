\documentclass{article}

\usepackage{listings}
\usepackage{amsmath}
\usepackage{graphicx}

\title{CS 325 HW1}
\date{1/13/15}
\author{Evan Steele}

\begin{document}

\lstset{language=bash}
\maketitle
\pagenumbering{gobble}
\newpage
\pagenumbering{arabic}

\section{Question 1}
The list of terms, in order from slowest growing to fastest growing
\begin{itemize}
   \item $n^{1/2}$
   \item $n log(n)$
   \item $(log(n)+1)^{3}$
   \item $n^{log_{3}(7)}$
   \item $2^{log(2)n}$, $5^{log(3)n}$
   \item $7^{2n}$
   \item $(1000(log(n)))^{3}$
\end{itemize}
\section{Question 2}
For each $f(n)$ and $g(n)$, it will be noted if the result is Big-O, Big-Theta, or Big-Omega
\begin{itemize}
   \item $3n+6$ , $10000n-500$ is Big-Theta
   \item $n^{1/2}$, $n^{1/3}$ is Big-O
   \item $log(7n)$, $log(n)$ is Big-Theta
   \item $n^{1.5}$, $nlog(n)$ is Big-Omega
   \item $\sqrt[2]{n}$, $(logn)^{3}$ is Big-Omega
   \item $n2^{n}$, $3^{n}$, is Big-O
\end{itemize}
\section{Question 3}
We know that the sum of a geometric series is the first term if it is increasing, or the last term if it is decreasing. We do have a closed-form for this series, which might look like this: $\frac{c^{n+1}-1}{c-1}$ for our series. With this, we can determine the remaining instances using limits. if $c<1$, we can take the limit as $n$ approaches infinity, which looks like this: $\lim_{x\to\infty} g(n) = \frac{1}{1-c}$ where we conclude that the time is constant. If $c=1$, we automatically get big-theta. If $c>1$, 
we have a new limit that looks like this: $\lim_{x\to\infty} g(n) = \frac{c-\frac{1}{c^{n}}}{c-1}$ for the function. This works out to $c^{c}$, which becomes our Big-O.
\section{Question 4}
We know, given the nature of the $n!$ function, that the expansion of the $log(n!)$ would look like $log(n!) = log(1) + log(2) + ... + log(n-1) + log(n)$, in which the upper bound for Big-O is given by $n*log(n)$. In order to get Big-Theta from this, we must also show that the growth for Big-Omega is the same. It works out to $n/2 * log(n/2)$ which gives us a Big-Omega of $n log(n)$
\end{document}
